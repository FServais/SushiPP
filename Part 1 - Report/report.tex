\documentclass[a4paper,titlepage]{article}

%PACKAGES
\usepackage[utf8]{inputenc}
\usepackage[T1]{fontenc}
\usepackage[english]{babel}
\usepackage{amsmath}
\usepackage{amssymb}
\usepackage{mathrsfs}
\usepackage{fancyhdr}
\usepackage{lmodern}
\usepackage{graphicx}
\usepackage{geometry}
\usepackage{fancybox}
\usepackage{textcomp}
\usepackage{epstopdf}

%Symbole euro
\usepackage{eurosym}

%Listings : affichage code
\usepackage{listings}

\geometry{hmargin=2.5cm}

%Elements de la page de garde
\begin{document}


\begin{titlepage}

\begin{figure}
\centering
\includegraphics[width=5cm]{logo-ulg.png}
\end{figure}



\title{
\vspace{0.2cm}
\LARGE{\textbf{Sushi++ - Language grammar and description}} \\ \textsc{Compilers}
\author{\textbf{Floriane Magera} \small{(S111295})\\\textbf{Romain Mormont} \small{(S110940})\\\textbf{Fabrice Servais} \small{(S111093})}\\
\date{March 6, 2015}
\rule{15cm}{1.5pt}
}

\end{titlepage}

%DOCUMENT
\pagestyle{fancy}
\lhead{Sushi++ - Language grammar and description}
\rhead{Compilers}

%Page de garde
\maketitle
  
\section{Introduction?}



\section{Description}

Sushi++ is a mostly functional language, enhanced with a few features belonging to the declarative and object-oriented paradigms.


  \subsection{Features}
Here is our list of features presented in a decreasing order of priority:
\begin{enumerate}
  \item Closure: a function will be able to capture a variable or a function from the scope of a higher function.
  \item Delimiter of end of line: $\backslash$n (...)
  \item Full type inference: 
  \item Usage of language C to use a few function: 
  \item Simple garbage collector: 
  \item Packages: 
\end{enumerate}


  \subsection{Keywords}



  \subsection{Datastructures}




  \subsection{Operators}
We decided to provide our language with the following operators :
\begin{itemize}
	\item The assignment operators : $ = $ for a variable, $ : $ for a function,
	\item Some arithmetic operators : $*, +, -, /, **, \%,$
	\item Some logic operators : $||, |, \&, \&\&, \verb+^+, !$
	\item Some comparison operators : $==, <, >, <=, >=,$
	\item Some special operators : incrementation : $++, --$ and concatenation : $.$, 
\end{itemize}


\newpage

\end{document}
