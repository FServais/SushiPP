\documentclass[a4paper,titlepage]{article}

%PACKAGES
\usepackage[utf8]{inputenc}
\usepackage[T1]{fontenc}
\usepackage[english]{babel}
\usepackage{amsmath}
\usepackage{amssymb}
\usepackage{mathrsfs}
\usepackage{fancyhdr}
\usepackage{lmodern}
\usepackage{graphicx}
\usepackage{geometry}
\usepackage{fancybox}
\usepackage{textcomp}
\usepackage{epstopdf}

%Symbole euro
\usepackage{eurosym}

%Listings : affichage code
\usepackage{listings}

\geometry{hmargin=2.5cm}

%Elements de la page de garde
\begin{document}


\begin{titlepage}

\begin{figure}
\centering
\includegraphics[width=5cm]{logo-ulg.png}
\end{figure}



\title{
\vspace{0.2cm}
\LARGE{\textbf{Sushi++ - Language grammar and description}} \\ \textsc{Compilers}
\author{\textbf{Floriane Magera} \small{(S111295})\\\textbf{Romain Mormont} \small{(S110940})\\\textbf{Fabrice Servais} \small{(S111093})}\\
\date{March 6, 2015}
\rule{15cm}{1.5pt}
}

\end{titlepage}

%DOCUMENT
\pagestyle{fancy}
\lhead{Sushi++ - Language grammar and description}
\rhead{Compilers}

%Page de garde
\maketitle
  
\section{Introduction?}



\section{Description}

Sushi++ is a mostly functional language, enhanced with a few features belonging to the imperative paradigm.


  \subsection{Features}
Here is our list of features presented in a decreasing order of priority:
\begin{enumerate}
  \item Anonymous function,
  \item Closure: a function will be able to capture a variable or a function from a higher scope,
  \item Delimiter of end of line: $\backslash$n (...),
  \item Full type inference,
  \item Type hinting: allow the user to specify the type of a function parameter,
  \item Built-in datastructures: array, list and tuple,
  \item Usage of language C to use a few function,
  \item Simple garbage collector: to clean the memory after the usage of built-in datastructures.
  \item Packages: functions could be declared in namespaces/packages.
\end{enumerate}


  \subsection{Keywords}

The keywords of this language are inspired from the lexical field of sushi food. We have:
\begin{itemize}
  \item \texttt{maki}: declaration of a variable or a named function.
  \item \texttt{soy}: declaration of an anonymous function.
  \item \texttt{roll}: while loop.
  \item \texttt{for}: for loop.
  \item \texttt{continue}, \texttt{break}: loop iteration control. 
  \item \texttt{if-elseif-else}: conditional structure.
  \item \texttt{menu}: switch structure.
  \item \texttt{nori}: return.
  \item \texttt{to}: list "constructor". 
  \item Types: \texttt{int}, \texttt{float}, \texttt{string}, \texttt{array}, \texttt{list}, \texttt{tuple}
  \item \texttt{mat}: define a package. This is not taken into account yet, but will be considered if time allows it.
\end{itemize}


  \subsection{Built-in datastructures}

We consider to implement 3 types of datastructure of which the behavior will be defined based on 3 properties:
\begin{itemize}
  \item \textbf{Structure mutability}: it is mutable if items can be added and removed after the creation of the datastructure,
  \item \textbf{Item mutability}: it is mutable if the items can be changed after the creation of the datastructure,
  \item \textbf{"Multityping"}: if the datastructure can contain elements of different types. 
\end{itemize}

\begin{table}[h!]
  \center
  \begin{tabular}{c|ccc}
    Name & Structure mutability & Item mutability & Multityping\\
    \hline
    \texttt{array} & Mutable & Mutable & Single type\\
    \texttt{list} & Immutable & Immutable & Single type\\
    \texttt{tuple} & Immutable & Mutable & Multiple types
  \end{tabular}
  \caption{Properties of the datastructures}
\end{table}

  \subsection{Operators}
We decided to provide our language with the following operators :
\begin{itemize}
  \item The assignment operators : $ = $ for a variable, $ : $ for a function,
  \item Some arithmetic operators : $*, +, -, /, **, \%,$
  \item Some logic operators : $||, |, \&, \&\&, \verb+^+, !$
  \item Some comparison operators : $==, <, >, <=, >=,$
  \item Some special operators : incrementation : $++, --$ and concatenation : $.$ 
\end{itemize}


\newpage

\end{document}
